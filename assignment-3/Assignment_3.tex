\documentclass[journal,12pt, two column]{IEEEtran}
\usepackage{mathtools}
\usepackage{enumitem}
\usepackage{graphicx}
\usepackage{tfrupee}
\usepackage{amssymb}
\usepackage{amsmath,amssymb,amsthm}
\let\vec\mathbf
\newcommand{\myvec}[1]{\ensuremath{\begin{pmatrix}#1\end{pmatrix}}}
\providecommand{\brak}[1]{\ensuremath{\left(#1\right)}}

\title{Assignment 3\\ \Large AI1110: Probability and Random Variables \\ \large Indian Institute of Technology Hyderabad}
\author{Shreyas Wankhede \\ \normalsize AI21BTECH11028 \\ \vspace*{20pt} \\ \Large CBSE Class 9 Statistics}

\begin{document}

\maketitle{\textbf{Example 15:}}\\
Consider a small unit of a factory where there are 5 employees:a supervisor and four labourers.The labourers draw a salary of \rupee 5,000 per month each while the supervisor \rupee 15,000 per month.Calculate the mean,median and mode of the salaries of this unit of the factory.

\textbf{Solution:}
For labourers have salary of \rupee5000 and supervisor has salary of \rupee15000.

\begin{enumerate}[label=(\roman*)]

\item
Mean
\begin{table}[!ht]
            \centering
            \resizebox{\columnwidth}{!}
            {
                \begin{tabular}{|c| c| }
                \hline
                 Salary($x_{i}$) & frequency($f_{i}$) \\
                 \hline
                 $5000$   & $4$ \\
                 $15000$   & $1$ \\
                 \hline
                    \textbf{Total} & $\sum_{i=1}^{2} f_{i} = 5$ \\
                 \hline
                \end{tabular}
            }
            \caption{}
            \label{table:table1}
       \end{table}
       
Let, \vec{X} be the column vector of salary ($x_{i}$), \vec{F} be the column vector of frequency ($f_{i}$), \vec{K} be the column matrix of 1's with 2 rows and \vec{m} be the mean\\
The formula for finding mean is:
\begin{equation}
 \label{eq:formulae}
 \vec{m} =\dfrac{\vec{X^{\top}}\vec{F}}{\vec{K^{\top}}\vec{F}} 
\end{equation}
Also,
\begin{align}
 \label{eq:formulae2}
	      	\vec{X}&=\myvec{5000,15000}   \\
	     \label{eq:formulae3}
	      	\vec{F}&=\myvec{4,1}    \\
	     \label{eq:formulae4}
      		\vec{K}&=\myvec{1,1}   
\end{align}
Putting the values of \eqref{eq:formulae2}, \eqref{eq:formulae3}, \eqref{eq:formulae4} in \eqref{eq:formulae},
\begin{align}
 \vec{m} = 7000
\end{align}
 $\therefore$ the mean of salaries is \rupee$7000$


\item
Median\\
To find median, we have to arrange the salaries in ascending order \\
$\implies 5000, 5000, 5000, 5000, 15000$ \\
 \textbf{Key Concept}: 
\begin{enumerate}
\item For a sorted data if number of observations($N$) is odd, then median of the data will be $\brak{\frac{N + 1}{2}}^{th}$ observation.
\item If the number of observations($N$) is even, then median will be the mean of $\brak{\frac{N}{2}}^{th}$ and $\brak{\frac{N+2}{2}}^{th}$ observations.
\end{enumerate}
As there are odd number of salaries(5),\\
Median = $\dfrac{5+1}{2}^{th}$ element = $3^{rd}$ element
\begin{align}
\implies median = 5000\nonumber
\end{align}
$\therefore$ the median of salaries is \rupee$5000$.

\item
Mode\\
In the data of salaries ,we can see that the salary data element $5000$ occurs maximum number of times (frequency = 4).
\begin{align}
  \implies mode=5000\nonumber
\end{align}
$\therefore$ the mode of salaries is \rupee$5000$.

\end{enumerate}
\end{document}
